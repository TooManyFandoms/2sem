\documentclass{article}
\usepackage{ucs} 
\usepackage[utf8x]{inputenc} 
\usepackage[russian]{babel} 
\title{\LaTeX}  
\date{} 
\usepackage{fancyhdr}
\usepackage{amssymb}
\usepackage{mathtools}
\renewcommand{\headrulewidth}{0pt}
\pagestyle{fancy}
\begin{document}
\large
\noindent пенью точности, т. е. будет выполнено неравенство (5.51). Это высказывание является, конечно, перефразировкой определения (5.50)---(5.51), разъясняющей интуитивное представление о непрерывной функции.

Так как непрерывность функции в точке является частным случаем существования предела функции, то определение непрерывности функции в точке можно дать в терминах окрестностей, надо лишь к условию (5.20) добавить требование $ x_0 \in X$~. Таким образом, функция $f$ , определенная на множестве $X$ , непрерывна в точке $x_0$, если для любой окрестности $U(y_0)$ точки $y_0 = f(x_0)$ существует такая окрестность $U(x_0)$ точки $x_0$, что выполняется включение $$ f(U(x_0)\cap X)\subset U(y_0),\quad x_0 \in X. \eqno (5.52) $$

Наконец, перенося постоянную $f(x_0)$ в равенстве (5.18) в левую часть, внося ее под знак предела и замечая, что обозначение $x\rightarrow x_0$ при пределе функции равносильно обозначению $x-x_0\rightarrow0$ (см. п. 5.4), получим $$ \lim_{x-x_0\to 0} [f(x)-f(x_0)]=0 \eqno (5.53)$$

Разность $x-x_0$ называется \emph{приращением аргумента} и обозначается через $\bigtriangleup x$, а разность $f(x)-f(x_0)$ --- \emph{приращением функции} $y=f(x)$, соответствующим данному приращению аргумента $\bigtriangleup x$, и обозначается через $\bigtriangleup y$. Таким образом, $$ \bigtriangleup x=x-x_0,\: \bigtriangleup y=f(x_0+\bigtriangleup x)-f(x_0), \: x_0\in X, \: x\in X. \eqno (5.54)$$
В этих обозначениях равенство (5.53) принимает вид $$ \lim_{\bigtriangleup x\to 0} \bigtriangleup y=0 \eqno (5.55) $$
т. е. непрерывность функции в точке означает, что бесконечно малому приращению аргумента соответствует бесконечно малое приращение функции.

Иногда бывает полезным условие непрерывности функции в точке, основанное на рассмотрении предела функции по проколотой окрестности.
\cfoot{\overline{\quad\quad\textsl{\thepage}\quad\quad}}
\setcounter{page}{199}
\newpage
\\* \textbf{\textsf{Л Е М М А 3.}} \emph{Если существует предел 
$$ \lim_{\substack{
x\to x_0 \\ 
x \in \mathring U(x_0) \cap X}} f(x)=a \eqno (5.56) $$ 
и функция $f$ определена в точке $x_0$, то $f$ непрерывна в точке $x_0$ тогда и только тогда, когда $f(x_0)=a$.}
\\*\textsf{Д о к а з а т е л ь с т в о.} Если функция $f$ непрерывна в точке $x_0$, т.~е. выполняется условие (5.13), то в силу очевидного включения $\mathring U(x_0) \cap X \subset X$ и того, что из существования предела по множеству следует существование предела и по любому подмножеству, имеем
$$\lim_{\substack {
x\to x_0 \\
x \in \mathring U(x_0) \cap X
}} f(x)=f(x_0). \eqno (5.57)$$
Из (5.56) и (5.57) следует, что $f(x_0)=a$.

Пусть теперь, наоборот, выполняется условие $f(x_0)=a$ и, следовательно,
$$\lim_{\substack {
x\to x_0 \\
x \in \mathring U(x_0) \cap X
}} f(x)\underset{(5.56)}{=} f(x_0). $$
Отсюда имеем, что для любой окрестности $U(f(x_0))$ точки $f(x_0)$ существует такая окрестность $U(x_0)$ точки $x_0$, что $$f(\mathring U(x_0)\cap X)\subset U(f(x_0)). \eqno (5.58)$$
Но, очевидно, $f(x_0)\in U(f(x_0))$ , поэтому в левой части включения (5.58) можно проколотую окрестность $\mathring U(x_0)$ заменить обычной окрестностью $U(x_0)$: $$f(U(x_0)\cap X)\subset U(f(x_0))$$
Это и означает, что функция $f$ непрерывна в точке $x_0$.\square

\textbf{\textsf{П р и м е р ы. 1.}} Функция $f(x)=c$, где $c$~---~постоянная, непрерывна на всей числовой прямой.

В самом деле, для любого $x_0\in \textbf{\textit{R}}$ имеет место равенство $$\lim_{x \to x_0} f(x)=\lim_{x \to x_0} c=c=f(x_0).\square$$

\textbf{\textsf{2.}} Функция $f(x)=\frac{1}{x}$ непрерывна в каждой точке $x_0\ne 0$.
В самом деле, $$\bigtriangleup y=f(x_0+\bigtriangleup x)-f(x_0)=\frac{1}{x+\bigtriangleup x}-\frac{1}{x_0}=-\frac{\bigtriangleup x_0}{(x_0+\bigtriangleup x)x_0},$$
\end{document}